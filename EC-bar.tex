%!TEX root=main.tex

% Drawing the Electoral College progress bar

%% Example usage:
%\leftbar{300}{Dfav};
%\leftbar{273}{Dwin};
%\rightbar{100}{Rfav};
%\rightbar{50}{Rwin};
%% These are on the main layer, so between the grey background and the foreground labels
%% Remember to draw each of the (longer) Xfav bars before the (shorter) Xwin bars


% Style for text labels
\tikzstyle{bartext}=[black, font=\huge]

% Parameters setting the left, right, bottom and top of the bar
%  The bar is 36 units wide, so each step corresponds to 15 EC votes
\def\barL{3}
\def\barR{39}
\def\barB{28}
\def\barT{30}

% Calculate the horizontal midpoint of the bar
\pgfmathparse{(\barL + \barR)/2};
\edef\barM{\pgfmathresult}

% Calculate the vertical midpoint of the bar
\pgfmathparse{(\barT + \barB)/2};
\edef\barC{\pgfmathresult}

% Draw the EC progress bar rectangle on the background layer;
%  This ensures that subsequent bars drawn on main are over this background
\begin{pgfonlayer}{background}
	\draw[wait, state] (\barL,\barB) rectangle (\barR,\barT);
\end{pgfonlayer}

% Draw and label the midpoint of the bar on the foreground layer
%  This ensures that the line is still visible when we draw bars on main
\begin{pgfonlayer}{foreground}
	\draw[ultra thick, black] (\barM,\barT) -- (\barM,\barB);
	\node[circle, black, fill=yellow, font=\huge] at (\barM,\barC) {270};
\end{pgfonlayer}



% Draw a rectangle from the left of the EC bar of length (#1/538) the bar
% Styled as #2, with label at position #3
\newcommand{\leftbar}[3]{
	\pgfmathparse{\barL + ((\barR - \barL)*(#1/538))};	
	\edef\barX{\pgfmathresult}
	
	\draw[state, #2] (\barL,\barT) rectangle (\barX,\barB);
	\node[bartext, below=#3mm, #2] at (\barX,\barB) {#1};
}


% Draw a rectangle from the right of the EC bar of length (#1/538) the bar
% Styled as #2, with label at position #3
\newcommand{\rightbar}[3]{
	\pgfmathparse{\barL + ((\barR - \barL)*(1 - #1/538))};	
	\edef\barX{\pgfmathresult}
	
	\draw[state, #2] (\barR,\barT) rectangle (\barX,\barB);
	\node[bartext, above=#3mm, #2] at (\barX,\barT) {#1};
}


% Now bundle these into commands for the 4 bars we want to draw

\newcommand{\Dfavbar}[1]{
	\leftbar{#1}{Dfav}{11};
}

\newcommand{\Dwinbar}[1]{
	\leftbar{#1}{Dwin}{1};
}

\newcommand{\Rfavbar}[1]{
	\rightbar{#1}{Rfav}{11};
}

\newcommand{\Rwinbar}[1]{
	\rightbar{#1}{Rwin}{1};
}

